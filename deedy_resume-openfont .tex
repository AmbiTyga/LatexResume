\documentclass[]{deedy-resume-openfont}


\begin{document}

%%%%%%%%%%%%%%%%%%%%%%%%%%%%%%%%%%%%%%
%
%     LAST UPDATED DATE
%
%%%%%%%%%%%%%%%%%%%%%%%%%%%%%%%%%%%%%%


%%%%%%%%%%%%%%%%%%%%%%%%%%%%%%%%%%%%%%
%
%     TITLE NAME
%
%%%%%%%%%%%%%%%%%%%%%%%%%%%%%%%%%%%%%%


\namesection{Ambesh}{Shekhar}{
\href{mailto:ambesh.sinha@gmail.com}{\Letter\ ambesh.sinha@gmail.com}\ \ \
\href{}{\faMobile\ +91-8825272821} \ \ \
\href{https://ambityga.github.io}{\faLink\ AmbiTyga}
\\
\href{https://www.github.com/AmbiTygya}{\faGithub\ Ambesh@Github}
\ \ \ \href{https://www.linkedin.com/in/ambesh-shekhar-1a115414b/}{\faLinkedinSquare\ Ambesh@LinkedIn} }\\

%%%%%%%%%%%%%%%%%%%%%%%%%%%%%%%%%%%%%%
%
%     COLUMN ONE
%
%%%%%%%%%%%%%%%%%%%%%%%%%%%%%%%%%%%%%%

\begin{minipage}[t]{0.33\textwidth} 

%%%%%%%%%%%%%%%%%%%%%%%%%%%%%%%%%%%%%%
%     EDUCATION
%%%%%%%%%%%%%%%%%%%%%%%%%%%%%%%%%%%%%%

\section{Education} 
\subsection{BIT Mesra} 
\descript{B.E. in Computer Science\\ and Engineering}
\location{Expected May 2020}

\sectionsep

%%%%%%%%%%%%%%%%%%%%%%%%%%%%%%%%%%%%%%
%     COURSEWORK
%%%%%%%%%%%%%%%%%%%%%%%%%%%%%%%%%%%%%%

\section{Coursework}


\subsection{Undergraduate}

Fundamental of Data Structure \\
Object Oriented Programming \\ 
Advanced Design and Analysis of Algorithms \\ 
Artificial Intelligence \\ 
Operating Systems \\ 
Database System \\ 
Computer Networks \\ 
Computer Structure and Architecture \\ 
Software Engineering\\
System Programming\\
Compiler Designs and Principles\\
Computer Graphics and Design\\
Software Project Management\\
\sectionsep

\subsection{Independent}
\href{https://online.codingblocks.com/app/certificates/CBOL-26635-2848}{Machine Learning Course (Coding Blocks)} \\
\href{http://cs229.stanford.edu/}{CS 229 Machine Learning} \\
\href{https://cs230.stanford.edu/}{CS 230 Deep Learning} \\
\href{https://coursera.org/share/0d85004d7396f79556aa1a92de8a3ab5}{Natural Language Processing Tensorflow (Coursera)} \\
\href{https://coursera.org/share/bf38126b8c3094913923da5075b257d5}{Improving Deep Learning Network(Coursera)} \\
Sequence Model(Coursera) \\
%%%%%%%%%%%%%%%%%%%%%%%%%%%%%%%%%%%%%%
%     SKILLS
%%%%%%%%%%%%%%%%%%%%%%%%%%%%%%%%%%%%%%

\section{Skills}
\subsection{Programming}
C/C++ \textbullet{}   Python \textbullet{} Java \\
\textbullet{} R \textbullet{} Android \textbullet{} Dart \\ 
\textbullet{} Javascript 
\sectionsep   \\ 
\subsection{scientific libraries}
Keras \textbullet{} Tensorflow \textbullet{} Pytorch \textbullet{} Scikit-learn \textbullet{} Pandas \textbullet{} Numpy \textbullet{} NLTK\\ \textbullet{} Librosa \textbullet{} OpenCV\\
\sectionsep
\subsection{SOFTWARE AND TOOLS}
PyCharm \textbullet{} Arduino \textbullet{} Raspbian
\textbullet{} Flutter\\ \textbullet{} Android \textbullet{} MATLAB \\
\sectionsep
\subsection{PERSONALITY SKILLS}
Leadership \textbullet{} Communication \\
\textbullet{} Management \textbullet{} Team Work\\
\sectionsep


%%%%%%%%%%%%%%%%%%%%%%%%%%%%%%%%%%%%%%
%
%     COLUMN TWO
%
%%%%%%%%%%%%%%%%%%%%%%%%%%%%%%%%%%%%%%

\end{minipage} 
\hfill
\begin{minipage}[t]{0.66\textwidth} 

%%%%%%%%%%%%%%%%%%%%%%%%%%%%%%%%%%%%%%
%     EXPERIENCE
%%%%%%%%%%%%%%%%%%%%%%%%%%%%%%%%%%%%%%

\section{Research Work}



\runsubsection{\href{https://bit.ly/2Yp1Hvk}{{\faPaperclip\ MemSem: A Multimodal framework for sentiment analysis}}}\\
\location{Feb 2020 - May 2020}
\text{MemSem is a neural network project which determines the sentiment analysis of posted memes.}\\
\vspace{\topsep}
\begin{tightemize}
\item Based on Multimodal neural network(Visual and Textual). \\
\item Works on images and OCR extracted text from memes. \\
\item Trained on multimodal network of VGG19 and BERT-based model. \\
\item Determines sentiment of memes. \\
\end{tightemize}
\sectionsep

\runsubsection{\href{https://bit.ly/2zmLONo}{{\faPaperclip\ QuesBELM: A BERT based Ensemble Language Model}}}\\
\location{March 2020 - May 2020}
QuesBELM is a natural question answering system that can help in answering to any queries. It uses ensemble methods and application of BERT models. \\
\begin{tightemize}
\item Based on Ensemble neural network architecture. \\
\item Comprises of BERT-base , BERT-large \ and ALBERT-XXL fine tuned on SQUaD. \\
\item Trained on Google's Natural Question Dataset which consists of queries from google and respective article to the query from wikipedia. \\
\item The system provides better results compared to its predecessor like \href{https://arxiv.org/pdf/1901.08634.pdf}{{A BERT Baseline for the Natural Questions}}
\end{tightemize}
\sectionsep



\section{Projects}

\runsubsection{\href{https://github.com/AmbiTyga/Pothole_Detection}{{\faGithub\ Pothole Detection}}}\\
Pothole Detection is full scale ML engine for real-time pothole detection.
Working on the con- currency issue in the RCNN and increasing the accuracy
of the output.\\

\begin{tightemize}
\item Based on Masked-RCNN. \\
\item Captures images using Raspberry-Pi and processes those images. \\
\item Predicts pothole in the way by using trained model deployed using AWS sagemaker. \\

\end{tightemize}
\sectionsep

\runsubsection{\href{https://github.com/AmbiTyga/ASL-Classifier}{{\faGithub\ ASL-Classifier}}}\\
A python script that utilizes the OpenCV and keras libraries to classify correct sign langugae\\

\begin{tightemize}
\item Uses images of American Sign Languages and ConvNet architecture. \\
\item Trained on preprocessed dataset and validated on self captured dataset. \\
\item Uses methods of OpenCV to create user interface to test on real-time dataset. \\
\end{tightemize}
\sectionsep

\runsubsection{\href{https://github.com/AmbiTyga/hiLyted}{{\faGithub\ hiLyted}}}\\
A video highlights creator, clips video from the given input by performing acoustics analysis\\

\begin{tightemize}
\item Downloads audio and video using youtube-dl.
\item Uses Librosa to extract audio data and sample rate. \\
\item Calculates and finds the right short time energy occurred in a 5 second window. \\
\item Clips the video of that duration and concatenate all the video into a single one using MoviePy. \\
\item Stores the highlighted video in local directory.
\end{tightemize}
\sectionsep



\end{minipage}
\newpage
\begin{minipage}[t]{0.33\textwidth}
%%%%%%%%%%%%%%%%%%%%%%%%%%%%%%%%%%%%%%
%     RESEARCH
%%%%%%%%%%%%%%%%%%%%%%%%%%%%%%%%%%%%%%

\subsection{Language}
\textbullet{Native} Hindi\\
\textbullet{Advanced} English\\
\textbullet{Basic} French\\

\section{Areas of Interest}
 Natural Language Processing \textbullet{} Computer Vision \textbullet{} Speech Analysis \textbullet{} Machine Learning \textbullet{} Robotics \textbullet{} Algorithms

\section{ACHIEVEMENTS} 
    \item {\href{https://drive.google.com/file/d/1YIRLa9Qt6k5Ax9tsddHiwMv6BL78XteQ/view?usp=sharing}{{\faTrophy\ \ 2nd Position in Internal Hackathon for SIH-2020}}}
\sectionsep
\end{minipage}
%%%%%%%%%%%%%%%%%%%%%%%%%%%%%%%%%%%%%%
%     AWARDS
%%%%%%%%%%%%%%%%%%%%%%%%%%%%%%%%%%%%%%
\hfill
\begin{minipage}[t]{0.66\textwidth}

\section{Ongoing Projects}
\runsubsection{Visual Question Answering System}\\
A python script, able to answer the question using COCO dataset from visualqa.org using the application of transfer learning and multi-modality.\\
\sectionsep

\runsubsection{Map of Characters}\\
A framework built with neo4j and python to describe the relations among characters in story given corpora using methods of Graph database, NER and NLP.\\
\sectionsep

\runsubsection{Social Distancing}\\
Surveillance on crowd to detect distance between individuals using UAV device and embedded electronics, and computer vision methods.
\sectionsep


\section{EXPERIENCE}
\runsubsection{Research Assistant}
\descript{| Birla Institute of Technology, Mesra}
\location{JAN 2020-Present}
\text{Working under Professor Smita Pallavi on the applications of deep learning and computer science}
\sectionsep


%%%%%%%%%%%%%%%%%%%%%%%%%%%%%%%%%%%%%%
%     SOCIETIES
%%%%%%%%%%%%%%%%%%%%%%%%%%%%%%%%%%%%%%



\end{minipage}
\end{document}