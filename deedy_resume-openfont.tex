%%%%%%%%%%%%%%%%%%%%%%%%%%%%%%%%%%%%%%%
% Deedy - One Page Two Column Resume
% LaTeX Template
% Version 1.1 (30/4/2014)
%
% Original author:
% Debarghya Das (http://debarghyadas.com)
%
% Original repository:
% https://github.com/deedydas/Deedy-Resume
%
% IMPORTANT: THIS TEMPLATE NEEDS TO BE COMPILED WITH XeLaTeX
%
% This template uses several fonts not included with Windows/Linux by
% default. If you get compilation errors saying a font is missing, find the line
% on which the font is used and either change it to a font included with your
% operating system or comment the line out to use the default font.
% 
%%%%%%%%%%%%%%%%%%%%%%%%%%%%%%%%%%%%%%
% 
% TODO:
% 1. Integrate biber/bibtex for article citation under publications.
% 2. Figure out a smoother way for the document to flow onto the next page.
% 3. Add styling information for a "Projects/Hacks" section.
% 4. Add location/address information
% 5. Merge OpenFont and MacFonts as a single sty with options.
% 
%%%%%%%%%%%%%%%%%%%%%%%%%%%%%%%%%%%%%%
%
% CHANGELOG:
% v1.1:
% 1. Fixed several compilation bugs with \renewcommand
% 2. Got Open-source fonts (Windows/Linux support)
% 3. Added Last Updated
% 4. Move Title styling into .sty
% 5. Commented .sty file.
%
%%%%%%%%%%%%%%%%%%%%%%%%%%%%%%%%%%%%%%%
%
% Known Issues:
% 1. Overflows onto second page if any column's contents are more than the
% vertical limit
% 2. Hacky space on the first bullet point on the second column.
%
%%%%%%%%%%%%%%%%%%%%%%%%%%%%%%%%%%%%%%

\documentclass[]{deedy-resume-openfont}

\begin{document}

%%%%%%%%%%%%%%%%%%%%%%%%%%%%%%%%%%%%%%
%
%     TITLE NAME
%
%%%%%%%%%%%%%%%%%%%%%%%%%%%%%%%%%%%%%%

\namesection{Ambesh}{Shekhar}{ \urlstyle{same}\url{https://github.com/AmbiTyga} \\
\href{mailto:ambesh.sinha@gmail.com}{ambesh.sinha@gmail.com} | +91-8825-2728-21
}

%%%%%%%%%%%%%%%%%%%%%%%%%%%%%%%%%%%%%%
%
%     COLUMN ONE
%
%%%%%%%%%%%%%%%%%%%%%%%%%%%%%%%%%%%%%%

\begin{minipage}[t]{0.33\textwidth} 

%%%%%%%%%%%%%%%%%%%%%%%%%%%%%%%%%%%%%%
%     EDUCATION
%%%%%%%%%%%%%%%%%%%%%%%%%%%%%%%%%%%%%%

%%%% 利用tikz来定位照片,部分招聘单位可能需要“以貌取人”
% \begin{tikzpicture}[remember picture, overlay] 
%   \node[anchor = north west] at ($(current page.north west)+(+1cm,-4cm)$) {\includegraphics[height=4.2cm]{620521198912304471}};
% \end{tikzpicture}%

% ~\\
% ~\\
% ~\\
% ~\\
% ~\\
% ~\\
% ~\\
% ~\\
% ~\\
% ~\\

\section{EDUCATION} 

\subsection{\href{www.bitmesra.ac.in}{BIT Mesra}}
\descript{B.E in Computer Science}
\descript{and Engineering}
\location{2017-Present}
\sectionsep

%%%%%%%%%%%%%%%%%%%%%%%%%%%%%%%%%%%%%%
%     LINKS
%%%%%%%%%%%%%%%%%%%%%%%%%%%%%%%%%%%%%%

% \section{链接} 
% GitHub://  \href{https://github.com/fuujiro}{\custombold{@fuujiro}} \\
% Blog://  \href{https://blog.fuujiro.com}{\custombold{fuujiro's island}} \\
% 知乎:// \href{https://www.zhihu.com/people/fuujiro/activities}{\custombold{fuujiro}} \\
% \sectionsep

%%%%%%%%%%%%%%%%%%%%%%%%%%%%%%%%%%%%%%
%     COURSEWORK
%%%%%%%%%%%%%%%%%%%%%%%%%%%%%%%%%%%%%%

\section{COURSEWORK}
\quad \\
\subsection{UNDERGRADUATE}

Fundamental of Data Structure \\
Object Oriented Programming \\ 
Advanced Design and Analysis of Algorithms \\ 
Artificial Intelligence \\ 
Operating Systems \\ 
Database System \\ 
Computer Networks \\ 
Computer Structure and Architecture \\ 

\sectionsep

\subsection{INDEPENDENT}

\href{https://online.codingblocks.com/app/certificates/CBOL-26635-2848}{Machine Learning Course (Coding Blocks)} \\
\href{http://cs229.stanford.edu/}{CS 229 Machine Learning} \\
\href{https://cs230.stanford.edu/}{CS 230 Deep Learning} \\
\href{https://coursera.org/share/0d85004d7396f79556aa1a92de8a3ab5}{Natural Language Processing Tensorflow (Coursera)} \\
Sequence Model(Coursera) \\
Improving Deep Learning Network(Coursera) \\
Sequences, Time Series and Predictions(Coursera) \\


%%%%%%%%%%%%%%%%%%%%%%%%%%%%%%%%%%%%%%
%     SKILLS
%%%%%%%%%%%%%%%%%%%%%%%%%%%%%%%%%%%%%%

\section{SKILLS}
\subsection{PROGRAMMING}
\textbullet{}   C/C++ \textbullet{}   Python \textbullet{} Java \\
\textbullet{} R \textbullet{} Android \textbullet{} Dart \\ 
\textbullet{} Javascript 
\sectionsep
\subsection{SCIENTIFIC LIBRARIES}
\textbullet{} Keras \textbullet{} Tensorflow \textbullet{} Pytorch \\
\textbullet{} Scikit-learn \textbullet{} Pandas \textbullet{} Numpy \\
\textbullet{} NLTK \textbullet{} Librosa \textbullet{} OpenCV \\ 
\sectionsep
\subsection{SOFTWARE AND TOOLS}
\textbullet{} PyCharm \textbullet{} Arduino \textbullet{} Raspbian \\
\textbullet{} Flutter \textbullet{} Android \textbullet{} MATLAB \\
\sectionsep
\subsection{PERSONALITY SKILLS}
\textbullet{} Leadership \textbullet{} Communication \\
\textbullet{} Management \textbullet{} Team Work
\sectionsep
\subsection{Language}
\textbullet{Native} Hindi\\
\textbullet{Advanced} English\\
\textbullet{Basic} French\\
%%%%%%%%%%%%%%%%%%%%%%%%%%%%%%%%%%%%%%
%
%     COLUMN TWO
%
%%%%%%%%%%%%%%%%%%%%%%%%%%%%%%%%%%%%%%

\end{minipage} 
\hfill
\begin{minipage}[t]{0.66\textwidth} 


%%%%%%%%%%%%%%%%%%%%%%%%%%%%%%%%%%%%%%
%     RESEARCH
%%%%%%%%%%%%%%%%%%%%%%%%%%%%%%%%%%%%%%

\section{PROJECTS}

\runsubsection{\href{https://colab.research.google.com/drive/1M54U4ifasZD6KcfdBkB7Opi3GhnRABxT}{{MemSem: A Multimodal framework for sentiment analysis}}}
\location{Feb 2020 - Present}
MemSem is a neural network project which determines the sentiment analysis
of posted memes. Working on improving the accuracy and training on huge
datasets. \\

\textbullet{} Based on Multimodal neural network(Visual and Textual). \\
\textbullet{} Works on images and OCR extracted text from memes. \\
\textbullet{} Trained on multimodal network of VGG19 and BERT-based model. \\
\textbullet{} Determines sentiment of memes. \\
\sectionsep

\runsubsection{\href{https://github.com/AmbiTyga/DeafDance}{{DeafDance}}}
\location{Jan 2020}
DeafDance is UI based script to understand sign-languages. Working on image
segmentation and calibrations of pixels in the images and trying pre-trained
models. \\

\textbullet{} Based on image processing and Computer vision. \\
\textbullet{} CNN trained on 2515 hand-sign images. \\
\textbullet{} Deployed with UI created using openCV. \\
\textbullet{} Predicts the gesture from webcam.\\
\sectionsep

\runsubsection{\href{https://github.com/AmbiTyga/hiLyted}{{hiLyted}}}
\location{Dec 2019}
hiLyted helps in creating highlights of matches from the given sources, covering
up the best moments of tournaments and games \\

\textbullet{} Based on Speech Analysis \\
\textbullet{} Processes audio from the video using Librosa. \\
\textbullet{} Clips match videos using moviepy. \\
\sectionsep

\runsubsection{\href{https://github.com/AmbiTyga/Pothole_Detection}{{Pothole Detection}}}
\location{June 2019}
Pothole_Detection is full scale ML engine for real-time pothole detection.
Working on the con- currency issue in the RCNN and increasing the accuracy
of the output.\\
\textbullet{} Based on Masked-RCNN. \\
\textbullet{} Captures images using Raspberry-Pi and processes those images. \\
\textbullet{} Predicts pothole in the way by using trained model imported from AWS sagemaker. \\
\sectionsep

% \runsubsection{\href{https://www.moberries.com/}{{MoBerries}}}
% \location{Jul 2016 – May 2017 | BuzzBlare, Lahore.}
% MoBerries is a revolutionized form of old-fashioned recruitment business. I have worked on its admin panel's front-end using Reactjs and Redux along with Material-UI for an amazing look and feel.
% \sectionsep

% \runsubsection{\href{https://playven.com/}{{Playven}}}
% \location{Jul 2016 – Nov 2016 | BuzzBlare, Lahore.}
% Playven is the easiest way to find your nearest sports venues and book a court. I worked on Reactjs and Redux version of this app.
% \sectionsep

% \runsubsection{\href{https://play.google.com/store/apps/details?id=com.abdulwasaetariq.odnvt}{{Optical Dictionary \& Vocabulary Teacher}}}
% \location{Aug 2015 – June 2016 | FAST-NUCES, Lahore.}
% This Mobile App provides a user with the ease of querying a dictionary. The user just has to focus the camera of his/her smart-phone over the word encountered and the application will show its’ meaning. I worked on its AI part, implementing an algorithm to classify letters.
% \sectionsep

%%%%%%%%%%%%%%%%%%%%%%%%%%%%%%%%%%%%%%
%     EXPERIENCE
%%%%%%%%%%%%%%%%%%%%%%%%%%%%%%%%%%%%%%

\section{EXTRA-CURRICULAR}

\runsubsection{Technical Club : Electronics and Communication } \\
\descript{| Secratary and Technical Support }
\location{Oct 2019 – Present}
\vspace{\topsep} % Hacky fix for awkward extra vertical space
\begin{tightemize}
\item Held Xordium. A technical festival for technology nerds presenting their projects.
\item Held a workshop on Competitive Programming and Machine learning workshop.
\end{tightemize}
\sectionsep

\runsubsection{Robonauts}
\descript{| Member }
\location{June 2018 - April 2019}
\begin{tightemize}
\item Member of robotics club of our college. Made IOT projects under guidance of seniors.
\item Took part in Techkriti19, made an IOT device for home automation.

\end{tightemize}
\sectionsep


%%%%%%%%%%%%%%%%%%%%%%%%%%%%%%%%%%%%%%
%     SOCIETIES
%%%%%%%%%%%%%%%%%%%%%%%%%%%%%%%%%%%%%%

\section{ACHIEVEMENTS} 

\textbullet{} \runsubsection{\href{https://drive.google.com/file/d/1YIRLa9Qt6k5Ax9tsddHiwMv6BL78XteQ/view?usp=sharing}{{2nd Position in Internal Hackathon for SIH-2020}}}
% \textbullet{} \runsubsection{\href{https://drive.google.com/file/d/1YIRLa9Qt6k5Ax9tsddHiwMv6BL78XteQ/view?usp=sharing}{{2nd Position in Internal Hackathon for SIH-2020}}}\\
% \textbullet{} Sports e.g. Cricket, Football\\
\sectionsep

\end{minipage} 
\end{document}